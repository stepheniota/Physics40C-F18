% LaTeX2e Template by Stephen Iota (https://stepheniota.github.io/)
% last updated: Aug. 2018

% for papers
%\documentclass[aps,onecolumn,superscriptaddress]{revtex4-1}

% https://www-d0.fnal.gov/Run2Physics/WWW/templates/revtex4.pdf
% https://cdn.journals.aps.org/files/revtex/auguide4-1.pdf
% for revTeX4-1 class options

% for other
\documentclass{article}
\usepackage[margin=1in]{geometry}

%%%%%%%%%%%%%%%%
%%% Packages %%%
%%%%%%%%%%%%%%%%

\usepackage[utf8]{inputenc}
%\usepackage{amsmath}
%\usepackage{amssymb}
%\usepackage{amsfonts}
%\usepackage{graphicx}
\usepackage[dvipsnames]{xcolor} % for colored links


% always put this at the end
\usepackage[
	colorlinks=true,
	citecolor=green!50!black,
	linkcolor=NavyBlue!75!black,
	urlcolor=green!50!black,
	hypertexnames=false]{hyperref} 

 
 %%%%%%%%%%%%%%%%%%
 %% New Commands %%
 %%%%%%%%%%%%%%%%%%
 
\newcommand{\email}[1]{\texttt{\href{mailto:#1}{#1}}}
 
%----------------------------------------------------
%%%%%%%%%%%%%%%%%%
%% Front Matter %%
%%%%%%%%%%%%%%%%%%

% no page numbers
\pagenumbering{gobble}


%%%%%%%%%%%%%
%%% Title %%%
%%%%%%%%%%%%%
\begin{document}

\begin{center}

\Large{\textsc{Worksheet 1}: \textbf{The $\vec{E}$ Field}}

\end{center}

\vspace{.5mm}

%%%%%%%%%%
%% INFO %%
%%%%%%%%%%

\begin{tabular}{rl}

\textsc{Course}:
&
Physics 40C (Fall 2018), Dr.~Laura Sales
\\
\textsc{SI Leader}
&
Stephen Iota (\email{siota001@ucr.edu})
\\
\textsc{Date}:
&
1st -- 5th of October 2018
\end{tabular}

%%%%%%%%%%%%%%
%% PROBLEMS %%
%%%%%%%%%%%%%%

\section{Understanding $\vec{E}$}

\begin{enumerate}

\item[(a)] Describe, in as much detail possible, what an electric field ($\vec{E}$) is. What are the units of the $\vec{E}$ field? What is the equation that governs the $\vec{E}$ field? Units?  What are some other examples of fields in physics? Similarities/differences? 






\item[(b)] An electron experiences a force of magnitude $F$ when it is $x$ distance away from a very long, charges wire with a linear charge density $\lambda$. If the charge density is doubled, at what distance from the wire will a proton experience a force of the same magnitude $F$? 

\end{enumerate}

\section{The Electric Dipole $\vec{p}$} 



\section{Charged Particle Field Interaction}

An electron is at the origin at $t = 0$ with an initial velocity $\vec{v} = 2\vec{x} + 1\vec{y}$ m/s, moving through a uniform electric field $\vec{E} = 3\vec{y}$ N/C. What is the electron's position at $t = 5$ s? 



\end{document}