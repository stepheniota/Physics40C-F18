% LaTeX2e Template by Stephen Iota (https://stepheniota.github.io/)
% last updated: Aug. 2018

% for papers
%\documentclass[aps,onecolumn,superscriptaddress]{revtex4-1}

% https://www-d0.fnal.gov/Run2Physics/WWW/templates/revtex4.pdf
% https://cdn.journals.aps.org/files/revtex/auguide4-1.pdf
% for revTeX4-1 class options

% for other
\documentclass[11pt]{article}
\usepackage[margin=1in]{geometry}

%%%%%%%%%%%%%%%%
%%% Packages %%%
%%%%%%%%%%%%%%%%

\usepackage[utf8]{inputenc}
\usepackage{amsmath}
\usepackage{amssymb}
\usepackage{amsfonts}
\usepackage{graphicx}
\usepackage[dvipsnames]{xcolor} % for colored links

%\usepackage{pdfpages}
%\makeatletter
%\AtBeginDocument{\let\LS@rot\@undefined}
%\makeatother


% always put this at the end
\usepackage[
	colorlinks=true,
	citecolor=green!50!black,
	linkcolor=NavyBlue!75!black,
	urlcolor=green!50!black,
	hypertexnames=false]{hyperref} 

 
%%%%%%%%%%%%%%%%
%%% Commands %%%
%%%%%%%%%%%%%%%%





%----------------------------------------------------
%%%%%%%%%%%%%%%%%%
%% Front Matter %%
%%%%%%%%%%%%%%%%%%

% no page numbers
\pagenumbering{gobble}


%%%%%%%%%%%%%
%%% Title %%%
%%%%%%%%%%%%%
\begin{document}

\begin{center}
\huge\textbf{{Physics 40C Fall 2018}}

\Large{\textsc{Supplemental Instruction Syllabus}} 


\end{center}




%%%%%%%%%%%%%%%%%%%%
%%%  Logistics   %%%
%%%%%%%%%%%%%%%%%%%%

\subsection*{Logistics}

\begin{tabular}{rl}

\textsc{SI Leader}:
&
Stephen Iota
\\
\textsc{Contact}:
&
\href{mailto:siota001@ucr.edu}{\texttt{siota001@ucr.edu}}
\\
\textsc{Webpage}:
&
\url{https://github.com/stepheniota/physics-40c-f18}
\\
\textsc{SI Sessions}:
&
\textbf{M 12 -- 1 pm Skye 112; TR 12 -- 1 pm Skye 108}
\\
\textsc{Lecture:}
&
Dr.~Laura Sales; TR 3:40 -- 5:00 pm Physics 2000
\\
\end{tabular}





%%%%%%%%%%%%%%%%%%%%%%%%%%
%% Official Description %%
%%%%%%%%%%%%%%%%%%%%%%%%%%


\subsection*{What is Supplemental Instruction?}

Supplemental Instruction \textsc{(SI)} is a free (!) academic support program that is designed to help students succeed in traditionally difficult courses. 
SI is held in a series of sessions throughout the quarter that are designed to review material in addition to lectures and discussions. 
Led by peer mentors, SI sessions combine ``what to learn'' with ``how to learn,''  and promote an active learning environment all in a small class setting.  
This gives students a unique learning experience that is more dynamic than the traditional (and boring) powerpoint slides lecture. 
It has been statistically shown that students who consistently participate in SI sessions receive \textbf{higher course grades and have higher graduation rates} vs.~those who never attend.\footnote{Erin M.~Buchanan, Kathrene D.~Valentine \& Michael L.~Frizell (2018) Supplemental Instruction: Understanding Academic Assistance in Underrepresented Groups, The Journal of Experimental Education, DOI: 10.1080/00220973.2017.1421517}


\subsection*{Session Goals}

Beyond the course material goals, the main goal of the \textsc{Phys040C} SI sessions to have students start to \emph{think like physicists.} In the SI sessions, [1] critical thinking, [2] problem solving and [3] fundamental concepts will be emphasized, in the hope of helping students not only in this specific class, but throughout their schooling and careers. 

\subsection*{Grades \& Participation}

SI is not a graded course. 
\textsc{Phys040C SI} sessions are all open, which means  attendance is entirely optional, but \emph{highly} recommended!
Students are encouraged to actively participate in order to learn and solidify course concepts.  


\subsection*{Guidlines}

We ask that students respect the classroom guidelines, as well as the SI Leader. 
This means: [1] always sign in before sessions, [2] no food or drink in classrooms, [3] no cell phone use during sessions and [4] no foul language or abusive behavior.
No hate speech will be tolerated and those guilty will be reported to the SI supervisor immediately. 

\begin{center}
	
{\bfseries I look forward to a great quarter with everyone! Feel free to contact me with any questions, I am here as an extra resource to help you succeed in Physics!}

\end{center}





\end{document}